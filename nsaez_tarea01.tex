\documentclass[a4paper,10pt]{article}
\usepackage[utf8x]{inputenc}
\usepackage{amsmath}
\usepackage{graphicx}
\usepackage[english]{babel}
\usepackage{url}
\usepackage{epstopdf}
\usepackage{subfig}
\usepackage{graphicx}

\title{Procesamiento Avanzado de Imágenes\\IEE3784}
\author{\textbf{Tarea 01}\\Norman F. S\'aez\\nfsaez@uc.cl}
\date{\today}

\begin{document}
\maketitle
\section{Pregunta 1}
%\begin{figure}[ht!]
%  \centering
%  \subfloat[Imagen Original]{\label{fig:210org}\includegraphics[width=0.35\textwidth]{img/original210.eps}}
%  ~ 
%  \subfloat[Imagen Ecualizada]{\label{fig:210heq}\includegraphics[width=0.35\textwidth]{img/histogEQ210.eps}}
%  ~ 
%  \subfloat[Imagen Filtro Laplaciano]{\label{fig:210lap}\includegraphics[width=0.35\textwidth]{img/laplace210.eps}}
%  ~ 
%  \caption{Imagen original, Ecualizada, y Filtro Laplaciano}
%  \label{fig:210img1_p1}
%\end{figure}

%Programe un software que construya B-splines cubicas no uniformes utilizando la recursion de Cox-de
%Boor vista en clase:
%Calcule el total de sumas y multiplicaciones necesarias para evaluar 100 puntos (por segmento) de
%la B-spline cubica (considere el caso mas favorable en terminos de operaciones e incluya las operaciones
%necesarias para construir las bases polinomiales).
%Para efectos del informe escrito, gra que en dos guras separadas la B-spline y las bases construidas
%con los siguientes puntos de control:
%y con el siguiente vector de nodos [1 3 4 7 7.5 10 20 25]

\end{document}
